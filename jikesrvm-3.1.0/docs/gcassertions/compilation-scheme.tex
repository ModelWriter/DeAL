\documentclass{report}
\usepackage{listings}

\newcommand{\Prod}{::=}
\newcommand{\VB}{\ |\ }
\newcommand{\tuple}[1]{\ensuremath{\langle #1 \rangle}}
\newcommand{\reachOp}{\ensuremath{\mathbf{R}}}
\newcommand{\reach}[1]{\ensuremath{\reachOp_{#1}}}
\newcommand{\disjointOp}{\ensuremath{\mathbf{X}}}
\newcommand{\disjoint}[2]{\ensuremath{{\disjointOp_{#1 + #2}}}}
\newcommand{\NT}[1]{\textsf{#1}}

\newcommand{\TTexclude}{\textsf{exclude-node}}
\newcommand{\TTinclude}{\textsf{include-node}}
\newcommand{\TTtid}{\textsf{mark-traversal-id}}
\newcommand{\TTset}{\textsf{set-predicates}}
\newcommand{\TTtrace}{\textsf{trace}}
\newcommand{\roots}{\ensuremath{\diamond}}

\begin{document}

\section{Heap invariant logic language}

\begin{verbatim}
Assertion ::= <QExpr> | <BoolExpr> | <Assertion> '||' <Assertion>
            | <Assertion> '&&' <Assertion> | '!' <Assertion>

Quant ::= 'forall' | 'exists'

QExpr ::= <Quant> <Type> <QName> '.' <BoolExpr>

BinOp ::= '==' | '!='

BoolExpr ::= <Expr> <BinOp> <Expr> | <Expr> 'instanceof' <Type>
           | <BoolExpr> '&&' <BoolExpr> | <BoolExpr> '||' <BoolExpr>
           | '!' <BoolExpr>
           | 'reach' '(' <ReachSpecSeq> ')' '[' <QName> ']'
           | 'disjoint' '(' <NodeSet> ',' <NodeSet> ')'

ReachSpecSeq ::= <ReachSpec> | <ReachSpecSeq> ',' <ReachSpec>

ReachSpec ::= <ReachSource> | <ReachSource> '/' <NodeSet>

ReachSource ::= 'root' | <NodeSet>

NodeSet ::= '{' <Name> */ ',' '}' | <Name>

Expr ::= <Name> | <QName> | <Expr> '.' <Name> | <int> | 'null'
\end{verbatim}

In the following, I will typeset the above `more nicely', using LaTeX:
\reachOp{} instead of \texttt{reach}, and \disjointOp{} instead of
\texttt{disjoint}.

\section{Trace and test language}

The trace-and-test language implements heap invariant logic on the
garbage collector side.  A trace-and-test program instructs the
garbage collector what to mark recursively, what to test for, and what
individual objects to expressly exclude or include in the traversal.

Each trace-and-test program runs in the context of a \emph{predicate
  family}.  A predicate family is an indexed family of predicates
(tests that apply on one object at a time) together with information
of whether the predicate family is existential or universal.
Predicate families correspond to \texttt{QExpr}s in the heap invariant
language.

The sweep algorithm applies one predicate from each family to each
heap object it encounters (modulo optimisations).

Below are the instructions for our trace-and-test language:

\begin{itemize}
  \item $\TTexclude(o)$: Exclude object $o$ from traversal.  This
    operation marks the object $o$ as `live,' meaning that
    they will be skipped during tracing.
  \item $\TTinclude(o)$: Mark object $o$ as `unknown' (grey), meaning
    that $o$ and its children will be traced the next time they are
    encountered.
  \item $\TTtid(x)$: Increase the traversal ID.  Each trace colours
    the objects it encounters with a particular traversal ID.  If a
    sweep encounters a live object with a traversal ID different from
    its own, it will flag $x$ as \textsf{false}.
  \item $\TTset(k_1, \ldots, k_n)$: Set the traversal predicates.
    This operation sets the particular predicate to test for each
    predicate family.
  \item $\TTtrace(o)$: Trace the heap, starting at object $o$.  $o$ may
    be the special node \roots, meaning all objects in the root set.
\end{itemize}


The default program is `$\TTtrace(\roots)$'.

A trace-and-test program is sound if (a) it includes a
$\TTtrace(\roots)$ and (b) for all $\TTexclude(o)$, there later is a
corresponding $\TTinclude(o)$ followed by $\TTtrace(o)$.

\subsection{Compilation}

To compile an assertion $A$ in heap invariant logic into the
trace-and-test language, we first label
each top-level \NT{QExpr} or \NT{BoolExpr}
as $i_k$ for some unique $k$. 

Let $A'$ assertion $A$ with $t_k$ replaced by $i_k$ such that
\[
 A'[i_{\overline{k}} \backslash t_{\overline{k}}] = A.
\]
Observe that each $t_k$ contains at most one quantifier, and if $t_k$
contains one or more `\reachOp{}' predicates, the `reach' must be over that
quantifier's variable.  `\reachOp{}' cannot appear outside of such
$t_k$.  We can thus partition our tests $t_k$ into the following:

\begin{enumerate}
\item boolean expressions without reachability/disjointedness predicates
\item disjointness predicates \disjointOp (possibly negated)
\item quantified $t_k$
\end{enumerate}

The first we can evaluate trivially.  Disjointedness tests we handle
in separation, by ensuring traversal non-overlap.
This leaves quantified $t_k$ of the form

\[
  Q x. P_k(x)
\]

where $Q \in \{ \exists, \forall \}$.

$P_k(x)$ is a predicate that may contain reachability tests based on
$x$.  $P_k$ may also contain $\disjointOp$ tests; we defer the
discussion of how we handle this case to
Section~\ref{sec:embedded-disjointness}.

Note that $P_k$ may contain different `reach' tests, as in $t_1$ below:

\[
  t_1 \equiv  \forall x. (\reach{a}(x) \wedge x.\textsf{v} = 0) \vee (\reach{b}(x) \wedge x.\textsf{v} = 1)
\]

  This construction is perfectly legitimate as long as we know that
`$a$' and `$b$' do not overlap (not knowing this is a static error.)

Each quantified predicate
$Q x. P_k(x)$
  we use an object that keeps track of whether P(x) was ever
satisfied or not satisfied.  However, the concrete test that we have
to run varies by traversal.  For example, in $t_1$ we expect different
values for $x.\textsf{v}$ depending on which traversal we encountered `\textsf{x}' in.
For each test $t_k$, we formalise this through a \emph{context predicate}

\[
  T_k(r)(x) \equiv P_k(x)[\reach{r}(x) \backslash \textsf{true}][\reach{-}(x) \backslash \textsf{false}]
\]

where $r \in R$ is one of the subscripts for $reachOp$ that occur in
the program.

We use the special form $T_k(\star)(x)$ to mean `replace all reachability
predicates by false' (i.e., `$\star$' is fresh).

  Note that each predicate may have any number of tests associated
with it.  For example,

\begin{eqnarray*}
  T_1(a)(x) &=& x.\textsf{v} \textsf{==} 0 \\
  T_1(b)(x) &=& x.\textsf{v} \textsf{==} 1 \\
  T_1(\star)(x) &=& \textsf{false}
\end{eqnarray*}

For each traversal, these context predicates capture our proof
obligation.  Note that none of the tests contains a reachability
predicate.
  Since the set of parameters for $r$ is finite, we can then
reify each of these tests into a predicate of the form

\begin{lstlisting}[language=Java]
  final static boolean
  test_k_r(Type x, Object[] environment)
\end{lstlisting}

Note that any result that is trivially `\textsf{true}' or
`\textsf{false}' cannot contribute to a universal test succeeding or
an existential test failing (respectively), so as a performance
optimisation we can omit these tests.  Some of the other tests may
have identical bodies, so we apply global value numbering to ensure
that we do not needlessly duplicate.

\subsection{Traversing the heap}

To emit trace-and-test code, we choose any order $r_1, \ldots, r_n$
over the elements of the set $R$ of indices for $\reachOp$ predicates
that occur in the program.  Observe that we require each of these to
be disjoint, so the order does not matter.

We thus first emit a $\TTtid(x)$ for a fresh $x$; this $x$ represents
the fact that we did not overlap any of the previous traversals.  In
the end, we can use the collection of all $x$ to assert our exclusion
properties.  In the end, the conjunction of all such $x$ guarantees
that our $\disjointOp$ predicates (occuring positively) hold.

Considering again our $r_i$, we observe that each has the form
\[
r_i = S_0 / I_0, \ldots, S_n / I_n
\]
where each $S_i$ is a root set, and each $I_i$ is an ignore set.
$S_i$ may be $\roots$, meaning that we must trace from the root set.

We begin the traversal with $\TTset(r_i)$, thereby informing each
predicate family $j$ about the particular traversal pass we are taking and
asking them to test $T_j(r_i)$, i.e., ensuring that for each predicate
call, $\reach{r_i}$ is \textsf{true} and $\reach{r_l}$ with $l \ne i$
is \textsf{false}.

Now, for each pair $S_j / I_j$ in ascending order, we first
$\TTexclude$ all elements of $I_j$ and then $\TTtrace$
all elements of $S_j$, in order.  If one of our ignore sets was
nonempty, this may leave part of the objects reachable from $\bigcup_j
S_j$ unreached.  We thus continue by $\TTinclude$ and $\TTtrace$ on
all objects in $\bigcup_j I_j$.

In the end, if we have not already done so, we $\TTtrace(\roots)$ to
ensure that the garbage collector fulfills its usual purpose, too.

\subsection{Epilogue}

Once the heap traversal is complete, we can fail any existential
predicate families that have been unable to observe any witness.  We
now have valuations for all $t_i$ (collected through our test families
$T_i$) and can evaluate the toplevel assertion.

\subsection{Embedded disjointness tests}\label{sec:embedded-disjointness}

Hmm.  We probably shouldn't allow $\disjointOp$ to occur in
\texttt{QExpr}.  So this won't be necessary.  (It is possible to `pull
them out' at least in some cases, though.)

\subsection{Run-time support}

All that we need for run-time support are mechanisms that track the
status of our predicate families (satisfied, unknown, unsatisfiable)
and indicators on each object on the heap to indicate the traversal
ID, to detect overlap.

\end{document}
